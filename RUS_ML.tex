%%%%%%%%%%%%%%%%%%%%%%%%%%%%%%%%%%%%%%%%%
% Compact Academic CV
% LaTeX Template
% Version 2.0 (6/7/2019)
%
% This template originates from:
% https://www.LaTeXTemplates.com
%
% Authors:
% Dario Taraborelli (http://nitens.org/taraborelli/home)
% Vel (vel@LaTeXTemplates.com)
%
% License:
% CC BY-NC-SA 3.0 (http://creativecommons.org/licenses/by-nc-sa/3.0/)
%
%%%%%%%%%%%%%%%%%%%%%%%%%%%%%%%%%%%%%%%%%

%----------------------------------------------------------------------------------------
%	PACKAGES AND OTHER DOCUMENT CONFIGURATIONS
%----------------------------------------------------------------------------------------

\documentclass[11pt]{article} % Default document font size

\input{structure.tex} % Include the file specifying the document structure and styling

% Set PDF meta-information
\hypersetup{
	pdftitle={Alexandra Shikunova - Curriculum vitae},
	pdfauthor={Alexandra Shikunova}
}
\usepackage{hyperref}
\usepackage{graphicx}
\usepackage{wrapfig}
\usepackage{paralist}
\usepackage{multicol}
\usepackage{pifont}

\graphicspath{ {./images/} }

%----------------------------------------------------------------------------------------

\begin{document}

%----------------------------------------------------------------------------------------
%	CONTACT AND GENERAL INFORMATION
%----------------------------------------------------------------------------------------

%\begin{figure} %this figure will be at the right
%    \centering
%    \includegraphics[width=0.22\textwidth]{photo2024}
%\end{figure}

{\LARGE\bfseries Александра Шикунова} % Name
\bigskip\bigskip\medskip % Whitespace

\begin{minipage}[t]{0.35\linewidth}
Москва, Россия\\
Год рождения: 2002\\
\end{minipage}
\begin{minipage}[t]{0.55\linewidth}
\textit{Контакты:}\\
\textbf{\href{https://t.me/thnlgrlivrlvdwsbrnwthrssnhrys}{Telegram}} \hspace*{2em}
\textbf{\href{mailto:notalexandrashikunova@gmail.com}{Email}} \hspace*{2em}
\textbf{\href{https://github.com/poisongrapevine}{Github}} \hspace*{2em}
\textbf{\href{https://thddbptnsndshs.github.io/academic_site/}{Сайт}} \hspace*{2em}
\textbf{\href{https://www.hse.ru/org/persons/401664223}{В Вышке}}
\medskip % Whitespace

\end{minipage}
%------------------------------------------------

\section*{Навыки}

\textbf{Языки}: Python, SQL, R\\
%\textbf{Что я умею}: NLP, LLM, диалоговые системы, инструментальная фонетика\\
\textbf{Библиотеки/фреймворки}: Huggingface, Pytorch, sklearn, pandas, numpy, nltk, PySpark, FastAPI, Flask, streamlit
%\vspace*{1em}
%\noindent\textbf{Что я умею:}
%
%	\vspace*{-1.5em}

%----------------------------------------------------------------------------------------
%	WORK EXPERIENCE
%----------------------------------------------------------------------------------------

\section*{Опыт работы}

\years{с 2023} \textbf{NLP-инженер, ПАО Сбербанк, центр ИИ блока ``Сервисы''}

%	\vspace*{-.75em}

\begin{minipage}[t]{0.95\linewidth}

    \begin{multicols}{2}
\begin{compactitem}[\ding{90}]
%	\item без LLM: классификация, NER, инфопоиск, дедубликация, суммаризация, regex
%	\item с LLM: SFT, instruction tuning, small LM, offline RL, prompt tuning
%	\item доведение моделей до прода, демо, работа со стажерами
	\item обучила SFT SLM с 400М параметров суммаризировать новости в 17 раз быстрее 7В модели при сравнимом качестве
	\item для ускорения обработки новостного потока дедублицировала, кластеризовала и суммаризировала новости
	\item построила единый пайплайн для задач инфопоиска, тестирующий несколько подходов (tf-idf, BM25, эмбеддеры)
	\item сводила задачи NER и поиска по реестру товаров к regex, над продуктивизацией поиска работала со стажером
\end{compactitem}
	\end{multicols}
	
\end{minipage}

	\vspace*{.5em}

\years{2022-2023} Аналитик больших данных, ПАО Мегафон, команда чат-бота

%	\vspace*{-.75em}
\begin{minipage}[t]{0.95\linewidth}

    \begin{multicols}{2}
\begin{compactitem}[\ding{165}]
	\item регулярное обновление продовых моделей intent \& entity recognition
	\item обернула модель кластеризации для разведки новых интентов в приложение для асессоров
\end{compactitem}
	\end{multicols}

\end{minipage}
	
	\vspace*{.5em}

\years{с 2021} \textbf{Стажер-исследователь в НУЛ по формальным моделям в лингвистике, ВШЭ}\\
\years{2020-2023} Преподавание олимпиадного английского в образовательном центре ``Взлёт''\\
\years{2023} Учебный ассистент по теории вероятности и математической статистике, Школа Лингвистики ВШЭ\\
\years{2022} Учебный ассистент по Python на майноре ``Интеллектуальный анализ данных'', ФКН ВШЭ\\
%\years{2022} Стажер направления ``Машинное обучение и анализ больших данных'', ПАО Мегафон, команда чат-бота\\
\years{2021} Учебный ассистент по дискретной математике, кафедра высшей математики ВШЭ

%----------------------------------------------------------------------------------------
%	EDUCATION
%----------------------------------------------------------------------------------------

\section*{Образование}

\years{2024-2026}\textbf{Магистратура ``Лингвистическая теория и описание языка'', ФГН НИУ ВШЭ, Москва}\\
\years{2020-2024}{Бакалаврская программа ``Фундаментальная и компьютерная лингвистика'', ФГН НИУ ВШЭ, Москва}\\
\years{2021-2023} Майнор ``Интеллектуальный анализ данных'', ФКН НИУ ВШЭ

\section*{Личные проекты}


\years{с 2021}Полевые исследования малых языков России: акустическая фонетика, морфосинтаксис\\
\years{2024}Обработка данных и построение инфографик для изучения фонологии \href{https://github.com/thddbptnsndshs/tolerance_principle_moksha}{мокшанского} и \href{https://github.com/thddbptnsndshs/forest_nenets_vowels/tree/develop}{лесного ненецкого}\\
%\years{2024}\href{https://github.com/thddbptnsndshs/arguementor}{Телеграм-бот для помощи в подготовке к ЕГЭ по русскому}\\
\years{2023-2024}\href{https://github.com/thddbptnsndshs/nonce-words}{NonceRNN -- генерация псевдослов с помощью RNN}\\
\years{2022}\href{https://github.com/thddbptnsndshs/MessageHDD}{MessageHDD -- метрика лексического разнообразия для корпуса маленьких текстов}\\
%\years{2022}\href{https://github.com/thddbptnsndshs/motustoday}{Инфографика по базе данных о мигрирующих животных Motus} (sql, sklearn, Flask)\\
\years{2021}\href{https://github.com/thddbptnsndshs/idioms}{Анализ данных синтаксического эксперимента на Python}
%\years{2021}\href{https://github.com/thddbptnsndshs/nlzlkskprdstvtljsclnhgrppncnvfrmvgrppvk}{Телеграм-бот, анализирующий орфографию записей в группах VK} (Telegram/VK API)\\

\section*{Языки}

%\textbf{Русский} родной\\
%\textbf{Английский} на уровне носителя, TOEFL C2 (2023)\\
%\textbf{Французский} средний уровень \\
%\textbf{Немецкий, персидский} на начальном уровне\\
%\textbf{Мокшанский, севернохантыйский, лесной ненецкий} полевая работа

\begin{minipage}[t]{0.95\linewidth}

    \begin{multicols}{2}
\begin{compactitem}[\ding{95}]
	\item \textbf{Русский} родной
	\item \textbf{Английский} TOEFL C2 (2023)
	\item \textbf{Французский} средний уровень
	\item \textbf{Немецкий, персидский} на начальном уровне
	\item \textbf{Мокшанский, севернохантыйский, лесной ненецкий} полевая работа
\end{compactitem}
	\end{multicols}

\end{minipage}

%----------------------------------------------------------------------------------------
%	FINAL FOOTER
%----------------------------------------------------------------------------------------

% Any final footer text such as a URL to the latest version of this CV, last updated date, compiled in XeTeX, etc

%----------------------------------------------------------------------------------------

\end{document}
