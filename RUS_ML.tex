%%%%%%%%%%%%%%%%%%%%%%%%%%%%%%%%%%%%%%%%%
% Compact Academic CV
% LaTeX Template
% Version 2.0 (6/7/2019)
%
% This template originates from:
% https://www.LaTeXTemplates.com
%
% Authors:
% Dario Taraborelli (http://nitens.org/taraborelli/home)
% Vel (vel@LaTeXTemplates.com)
%
% License:
% CC BY-NC-SA 3.0 (http://creativecommons.org/licenses/by-nc-sa/3.0/)
%
%%%%%%%%%%%%%%%%%%%%%%%%%%%%%%%%%%%%%%%%%

%----------------------------------------------------------------------------------------
%	PACKAGES AND OTHER DOCUMENT CONFIGURATIONS
%----------------------------------------------------------------------------------------

\documentclass[11pt]{article} % Default document font size

%%%%%%%%%%%%%%%%%%%%%%%%%%%%%%%%%%%%%%%%%
% Compact Academic CV
% Structural Definitions
% Version 1.0 (6/7/2019)
%
% This template originates from:
% https://www.LaTeXTemplates.com
%
% Authors:
% Dario Taraborelli (http://nitens.org/taraborelli/home)
% Vel (vel@LaTeXTemplates.com)
%
% License:
% CC BY-NC-SA 3.0 (http://creativecommons.org/licenses/by-nc-sa/3.0/)
%
%%%%%%%%%%%%%%%%%%%%%%%%%%%%%%%%%%%%%%%%%

%----------------------------------------------------------------------------------------
%	REQUIRED PACKAGES AND MISC CONFIGURATIONS
%----------------------------------------------------------------------------------------

\usepackage{graphicx} % Required for including images

\setlength{\parindent}{0pt} % Stop paragraph indentation

%----------------------------------------------------------------------------------------
%	MARGINS
%----------------------------------------------------------------------------------------

\usepackage{geometry} % Required for adjusting page dimensions and margins

\geometry{
	paper=a4paper, % Paper size, change to letterpaper for US letter size
	top=3.25cm, % Top margin
	bottom=4cm, % Bottom margin
	left=2.5cm, % Left margin
	right=2cm, % Right margin
	headheight=0.75cm, % Header height
	footskip=1cm, % Space from the bottom margin to the baseline of the footer
	headsep=0.75cm, % Space from the top margin to the baseline of the header
	%showframe, % Uncomment to show how the type block is set on the page
}

%----------------------------------------------------------------------------------------
%	FONTS
%----------------------------------------------------------------------------------------

\usepackage[utf8]{inputenc} % Required for inputting international characters
\usepackage[T1]{fontenc} % Output font encoding for international characters

\usepackage[semibold]{ebgaramond} % Use the EB Garamond font with a reduced bold weight

%----------------------------------------------------------------------------------------
%	SECTION STYLING
%----------------------------------------------------------------------------------------

\usepackage{sectsty} % Allows changing the font options for sections in a document

\sectionfont{\fontsize{13.5pt}{18pt}\selectfont} % Set font options for sections
\subsectionfont{\mdseries\scshape\normalsize} % Set font options for subsections
\subsubsectionfont{\mdseries\upshape\bfseries\normalsize} % Set font options for subsubsections

%----------------------------------------------------------------------------------------
%	MARGIN YEARS
%----------------------------------------------------------------------------------------

\usepackage{marginnote} % Required to output text in the margin

\newcommand{\years}[1]{\marginnote{\scriptsize #1}} % New command for adding years to the margin
\renewcommand*{\raggedleftmarginnote}{} % Left-align the years in the margin
\setlength{\marginparsep}{-10pt} % Move the margin content closer to the text
\reversemarginpar % Margin text to be output into the left margin instead of the default right margin

%----------------------------------------------------------------------------------------
%	COLOURS
%----------------------------------------------------------------------------------------

\usepackage[usenames, dvipsnames]{xcolor} % Required for specifying colours by name

%----------------------------------------------------------------------------------------
%	LINKS
%----------------------------------------------------------------------------------------

\usepackage[bookmarks, colorlinks, breaklinks]{hyperref} % Required for links

% Set link colours
\hypersetup{
	linkcolor=blue,
	citecolor=blue,
	filecolor=black,
	urlcolor=MidnightBlue
}
 % Include the file specifying the document structure and styling

% Set PDF meta-information
\hypersetup{
	pdftitle={Alexandra Shikunova - Curriculum vitae},
	pdfauthor={Alexandra Shikunova}
}
\usepackage{hyperref}
\usepackage{graphicx}
\usepackage{wrapfig}
\graphicspath{ {./images/} }

%----------------------------------------------------------------------------------------

\begin{document}

%----------------------------------------------------------------------------------------
%	CONTACT AND GENERAL INFORMATION
%----------------------------------------------------------------------------------------

%\begin{figure} %this figure will be at the right
%    \centering
%    \includegraphics[width=0.22\textwidth]{photo2024}
%\end{figure}

{\LARGE\bfseries Александра Шикунова} % Name
\bigskip\bigskip\medskip % Whitespace

Москва, Россия\\
Год рождения: 2002\\

\textit{Контакты:}\\
\textbf{\href{https://t.me/thnlgrlivrlvdwsbrnwthrssnhrys}{Telegram}} \hspace*{2em}
\textbf{\href{mailto:notalexandrashikunova@gmail.com}{Email}} \hspace*{2em}
\textbf{\href{https://github.com/poisongrapevine}{Github}} \hspace*{2em}
\textbf{\href{https://thddbptnsndshs.github.io/academic_site/}{Личный сайт}} \hspace*{2em}
\textbf{\href{https://www.hse.ru/org/persons/401664223}{В Вышке}}
\medskip % Whitespace

%------------------------------------------------

\section*{Навыки и интересы}

\textbf{Технические навыки}: Python, SQL, LaTeX, JavaScript (для PCIbex)\\
\textbf{Интересы}: LLM, диалоговые системы, prompt recovery, теоретическая лингвистика, пробинг, RL

%----------------------------------------------------------------------------------------
%	EDUCATION
%----------------------------------------------------------------------------------------

\section*{Образование}

\years{2020-2024}\textbf{Магистратура ``Лингвистическая теория и описание языка'', ФГН НИУ ВШЭ, Москва}\\
\years{2020-2024}{Бакалаврская программа ``Фундаментальная и компьютерная лингвистика'', ФГН НИУ ВШЭ, Москва}\\
\years{2021-2023} Майнор ``Интеллектуальный анализ данных'', ФКН НИУ ВШЭ

%----------------------------------------------------------------------------------------
%	WORK EXPERIENCE
%----------------------------------------------------------------------------------------

\section*{Опыт работы}

\years{с 2023} \textbf{NLP-инженер в центре ИИ блока ``Сервисы'' ПАО Сбербанк}\\
\years{с 2021} \textbf{Стажер-исследователь в НУЛ по формальным моделям в лингвистике, ВШЭ}\\
\years{2022-2023} Аналитик больших данных, ПАО Мегафон, команда чат-бота\\
\years{2020-2023} Преподавание олимпиадного английского в образовательном центре ``Взлёт''\\
\years{2023} Учебный ассистент по теории вероятности и математической статистике, Школа Лингвистики ВШЭ\\
\years{2022} Учебный ассистент по Python на майноре ``Интеллектуальный анализ данных'', ФКН ВШЭ\\
\years{2022} Стажер направления ``Машинное обучение и анализ больших данных'', ПАО Мегафон, команда чат-бота\\
\years{2021} Учебный ассистент по дискретной математике, кафедра высшей математики ВШЭ

%\section*{Доклады и публикации}

%\years{доклад} \textbf{ТМП 2022} Idioms, NP position and control: evidence from Russian (в соавторстве с Павлом Рудневым)\\
%\years{доклад} \textbf{SOUL 4} Case and agreement puzzle in the Moksha debitive\\
%\years{постер, статья} \textbf{ConSOLE 30} Mermaid construction: a case of Kazym Khanty\\

\section*{Личные проекты}

\years{2024}Обработка данных и построение инфографик для изучения фонологии \href{https://github.com/thddbptnsndshs/tolerance_principle_moksha}{мокшанского} \href{https://github.com/thddbptnsndshs/forest_nenets_vowels/tree/develop}{лесного ненецкого}\\
%\years{2024}\href{https://github.com/thddbptnsndshs/arguementor}{Телеграм-бот для помощи в подготовке к ЕГЭ по русскому}\\
\years{2023-2024}\href{https://github.com/thddbptnsndshs/nonce-words}{NonceRNN -- генерация псевдослов с помощью RNN}\\
\years{2023-2024}\href{https://github.com/thddbptnsndshs/texops}{TeXOps} и \href{https://github.com/thddbptnsndshs/atambibki}{atambibki} -- мои инструменты для создания проектов и управления библиографией в LaTeX\\
\years{2022}\href{https://github.com/thddbptnsndshs/MessageHDD}{MessageHDD -- метрика лексического разнообразия для корпуса маленьких текстов}\\
%\years{2022}\href{https://github.com/thddbptnsndshs/motustoday}{Инфографика по базе данных о мигрирующих животных Motus} (sql, sklearn, Flask)\\
\years{2021}\href{https://github.com/thddbptnsndshs/idioms}{Анализ данных синтаксического эксперимента с помощью Python}\\
%\years{2021}\href{https://github.com/thddbptnsndshs/nlzlkskprdstvtljsclnhgrppncnvfrmvgrppvk}{Телеграм-бот, анализирующий орфографию записей в группах VK} (Telegram/VK API)\\

\section*{Языки}

\textbf{Русский} родной\\
\textbf{Английский} на уровне носителя\\
\textbf{Французский} средний уровень \\
\textbf{Немецкий, персидский} на начальном уровне


%----------------------------------------------------------------------------------------
%	FINAL FOOTER
%----------------------------------------------------------------------------------------

% Any final footer text such as a URL to the latest version of this CV, last updated date, compiled in XeTeX, etc

%----------------------------------------------------------------------------------------

\end{document}
